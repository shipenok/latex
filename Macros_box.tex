%%%%%%%%%%%%%%%%%%%%%%%%%%%%%%%%%%%%%%%%%%%%%%%%%%%%%%%%%%%%%
\newcommand{\mycomment}[1]{}
%%%%%%%%%%%%%%%%%%%%%%%%%%%%%%%%%%%%%%%%%%%%%%%%%%%%%%%%%%%%%
\setlength{\parindent}{1cm}

\newcommand{\boxi}[4]{%
\newtcbtheorem[auto counter, number within = chapter]
{#1}{#2}{%                                                        
  breakable,
  fonttitle = \bfseries,
  colframe = #3,
  colback = #4
}{th}
}
%%%%%%%%%%%%%%%%%%%%%%%%%%%%%%%%%%%%%%%%%%%%%%%%%%%%%%%%%%%%%
\newcommand{\boxii}[4]{%
\tcbuselibrary{theorems,skins,hooks}
\newtcbtheorem[number within=chapter]{#1}{#2}
{%
	enhanced,
	breakable,
	colback = #3,
	frame hidden,
	boxrule = 0sp,
	borderline west = {2pt}{0pt}{#4},
	sharp corners,
	detach title,
	before upper = \tcbtitle\par\smallskip,
	coltitle = #4,
	fonttitle = \bfseries\sffamily,
	description font = \mdseries,
	separator sign none,
	segmentation style={solid, #4},
}
{qt}
}
%%%%%%%%%%%%%%%%%%%%%%%%%%%%%%%%%%%%%%%%%%%%%%%%%%%%%%%%%%%%%
\newcommand{\boxiii}[4]{%
\newtcbtheorem[number within=chapter]{#1}{#2}
{%
	colback = #3
	,breakable
	,colframe = #4
	,coltitle = #4
	,boxrule = 1pt
	,sharp corners
	,detach title
	,before upper=\tcbtitle\par\smallskip
	,fonttitle = \bfseries
	,description font = \mdseries
	,separator sign none
	,description delimiters parenthesis
}
{ex}
}
%%%%%%%%%%%%%%%%%%%%%%%%%%%%%%%%%%%%%%%%%%%%%%%%%%%%%%%%%%%%%
\usepackage{varwidth}
%
\newcommand{\boxiv}[6]{
\newtcbtheorem[number within=chapter]{#1}{#2}{enhanced,
	before skip=2mm,after skip=2mm, 
	colback=#3,
	colframe=#4,
	boxrule=0.5mm,
	attach boxed title to top left={xshift=1cm,yshift*=1mm-\tcboxedtitleheight}, varwidth boxed title*=-3cm,
	boxed title style={frame code={
					\path[fill=tcbcolback]
					([yshift=-1mm,xshift=-1mm]frame.north west)
					arc[start angle=0,end angle=180,radius=1mm]
					([yshift=-1mm,xshift=1mm]frame.north east)
					arc[start angle=180,end angle=0,radius=1mm];
					\path[left color=#5,right color=#5,
						middle color=#6]
					([xshift=-2mm]frame.north west) -- ([xshift=2mm]frame.north east)
					[rounded corners=1mm]-- ([xshift=1mm,yshift=-1mm]frame.north east)
					-- (frame.south east) -- (frame.south west)
					-- ([xshift=-1mm,yshift=-1mm]frame.north west)
					[sharp corners]-- cycle;
				},interior engine=empty,
		},
	fonttitle=\bfseries,
	%title={#2},#1
	}{def}}
%%%%%%%%%%%%%%%%%%%%%%%%%%%%%%%%%%%%%%%%%%%%%%%%%%%%%%%%%%%%%
\newcommand{\boxv}[4]{%
\newtcbtheorem{#1}{#2}{enhanced,
	breakable,
	colback=#3,
	colframe=#4,
	attach boxed title to top left={yshift*=-\tcboxedtitleheight},
	fonttitle=\bfseries,
	%title={#2},
	boxed title size=title,
	boxed title style={%
			sharp corners,
			rounded corners=northwest,
			colback=tcbcolframe,
			boxrule=0pt,
		},
	underlay boxed title={%
			\path[fill=tcbcolframe] (title.south west)--(title.south east)
			to[out=0, in=180] ([xshift=5mm]title.east)--
			(title.center-|frame.east)
			[rounded corners=\kvtcb@arc] |-
			(frame.north) -| cycle;
		},
	%#1
}{tn}
}
%%%%%%%%%%%%%%%%%%%%%%%%%%%%%%%%%%%%%%%%%%%%%%%%%%%%%%%%%%%%%
\newcommand{\boxvi}[4]{%
\newtcbtheorem[number within=chapter]{#1}{#2}{
	breakable,
	enhanced,
	colback=#3,
	colframe=#4,
	arc=0pt,
	outer arc=0pt,
	fonttitle=\bfseries\sffamily,
	colbacktitle=#4,
	attach boxed title to top left={},
	boxed title style={
			enhanced,
			skin=enhancedfirst jigsaw,
			arc=3pt,
			bottom=0pt,
			interior style={fill=#4}
		},
	%#1
}{str}
}
%%%%%%%%%%%%%%%%%%%%%%%%%%%%%%%%%%%%%%%%%%%%%%%%%%%%%%%%%%%%%
\newcommand{\boxvii}[7]{%
\usetikzlibrary{arrows,calc,shadows.blur}
\tcbuselibrary{skins}
\newtcolorbox{#1}[1][]{%
	enhanced jigsaw,
	colback=#3,%
	colframe=#4,
	size=small,
	boxrule=1pt,
	title=\textbf{#2:},
	halign title=flush center,
	coltitle=black,
	breakable,
	drop shadow=#5,
	attach boxed title to top left={xshift=1cm,yshift=-\tcboxedtitleheight/2,yshifttext=-\tcboxedtitleheight/2},
	minipage boxed title=#6,
	boxed title style={%
			colback=#7,
			size=fbox,
			boxrule=1pt,
			boxsep=2pt,
			underlay={%
					\coordinate (dotA) at ($(interior.west) + (-0.5pt,0)$);
					\coordinate (dotB) at ($(interior.east) + (0.5pt,0)$);
					\begin{scope}
						\clip (interior.north west) rectangle ([xshift=3ex]interior.east);
						\filldraw [white, blur shadow={shadow opacity=60, shadow yshift=-.75ex}, rounded corners=2pt] (interior.north west) rectangle (interior.south east);
					\end{scope}
					\begin{scope}[#4]
						\fill (dotA) circle (2pt);
						\fill (dotB) circle (2pt);
					\end{scope}
				},
		},
	%#1,
}
}
%%%%%%%%%%%%%%%%%%%%%%%%%%%%%%%%%%%%%%%%%%%%%%%%%%%%%%%%%%%%%
\newcommand{\boxviii}[2]{\setlength{\parindent}{0cm}\textbf{\textit{#1.}\\}\setlength{\parindent}{1cm}\begin{small}
#2
\end{small}}
%%%%%%%%%%%%%%%%%%%%%%%%%%%%%%%%%%%%%%%%%%%%%%%%%%%%%%%%%%%%%
\boxii{Theorem}{Теорема}{mylenmabg}{mylenmafr}
\newcommand{\thm}[3]{\begin{Theorem}{#1}{#3}#2\end{Theorem}}
\crefformat{tcb@cnt@Theorem}{~#2#1#3}
% 
\boxii{Proposition}{Предложение}{mylenmabg}{mylenmafr}
\newcommand{\mprop}[3]{\begin{Proposition}{#1}{#3}#2\end{Proposition}}
\crefformat{tcb@cnt@Proposition}{~#2#1#3}
% 
\boxii{Lemma}{Лемма}{mylenmabg}{mylenmafr}
\newcommand{\lemma}[3]{\begin{Lemma}{#1}{#3}#2\end{Lemma}}
\crefformat{tcb@cnt@Lemma}{~#2#1#3}
% 
\boxii{Corollary}{Следствие}{mylenmabg}{mylenmafr}
\newcommand{\mcor}[3]{\begin{Corollary}{#1}{#3}#2\end{Corollary}}
\crefformat{tcb@cnt@Corollary}{~#2#1#3}
% 
\boxii{Claim}{Утверждение}{mylenmabg}{mylenmafr}
\newcommand{\clm}[3]{\begin{Claim}{#1}{#3}#2\end{Claim}}
\crefformat{tcb@cnt@Claim}{~#2#1#3}
% 
\boxii{Definition}{Определение}{myg!10}{myg}
\newcommand{\dfn}[3]{\begin{Definition}{#1}{#3}#2\end{Definition}}
\crefformat{tcb@cnt@Definition}{~#2#1#3}
%
\boxii{Example}{Пример}{mytheorembg}{mytheoremfr}
\newcommand{\exmpl}[3]{\begin{Example}{#1}{#3}#2\end{Example}}
\crefformat{tcb@cnt@Example}{~#2#1#3}
%
\boxii{Exercise}{Упражнение}{red!5}{red!80!black}
\newcommand{\exsz}[3]{\begin{Exercise}{#1}{#3}#2\end{Exercise}}
\crefformat{tcb@cnt@Exercise}{~#2#1#3}
% 
\boxii{Note}{Замечание}{yellow!20!white}{yellow!80!black}
\newcommand{\nt}[3]{\begin{Note}{#1}{#3}#2\end{Note}}
\crefformat{tcb@cnt@Note}{~#2#1#3}
%%%%%%%%%%%%%%%%%%%%%%%%%%%%%%%%%%%%%%%%%%%%%%%%%%%%%%%%%%%%%
\newcommand{\solve}[1]{\boxviii{Решение}{#1}}
\newcommand{\prooff}[1]{\boxviii{Доказательство}{#1}}
%%%%%%%%%%%%%%%%%%%%%%%%%%%%%%%%%%%%%%%%%%%%%%%%%%%%%%%%%%%%%

% наработки
\mycomment{
\boxi{ThRule}{Правило}{mylenmafr}{mylenmabg}
\newcommand{\mrul}[3]{\begin{ThRule}{#1}{#3}#2\end{ThRule}}
\crefformat{tcb@cnt@ThRule}{~#2#1#3}
%
%
\boxi{WConc}{Неправильная концепция}{myr}{white}%
\newcommand{\wc}[3]{\begin{WConc}{#1}{#3}#2\end{WConc}}
\crefformat{tcb@cnt@WConc}{~#2#1#3}
%
%
\boxi{Hypothesis}{Гипотеза}{blue!75!black}{blue!10}
\newcommand{\mhyp}[3]{\begin{Hypothesis}{#1}{#3}#2\end{Hypothesis}}
\crefformat{tcb@cnt@Hypothesis}{~#2#1#3}
%
%
\boxi{Explanation}{Объяснение}{myg}{myg!10}%
\newcommand{\mexpln}[3]{\begin{Explanation}{#1}{#3}#2\end{Explanation}}
\crefformat{tcb@cnt@Explanation}{~#2#1#3}
%
%%
%
\boxii{Citatai}{Цитата}{mytheorembg}{mytheoremfr}
\newcommand{\citi}[3]{\begin{Citatai}{#1}{#3}#2\end{Citatai}}
\crefformat{tcb@cnt@Citatai}{~#2#1#3}
%
\boxii{Citataii}{Цитата преподавателя}{mylenmabg}{mylenmafr}
\newcommand{\citii}[3]{\begin{Citataii}{#1}{#3}#2\end{Citataii}}
\crefformat{tcb@cnt@Citataii}{~#2#1#3}
%
\boxii{Citataiii}{Сильная мысль}{myg!10}{myg}%
\newcommand{\citiii}[3]{\begin{Citataiii}{#1}{#3}#2\end{Citataiii}}
\crefformat{tcb@cnt@Citataiii}{~#2#1#3}
%
%%
%
\boxiii{Example}{Пример}{myexamplebg}{myg}
\newcommand{\exmpl}[3]{\begin{Example}{#1}{#3}#2\end{Example}}
\crefformat{tcb@cnt@Example}{~#2#1#3}
%
\boxiii{Exercise}{Упражнение}{red!5}{red!80!black}
\newcommand{\exsz}[3]{\begin{Exercise}{#1}{#3}#2\end{Exercise}}
\crefformat{tcb@cnt@Exercise}{~#2#1#3}
%
%%
%
\boxiv{Definition}{Определение}{red!5}{red!80!black}{tcbcolback!60!black}{tcbcolback!80!black}
\newcommand{\dfn}[3]{\begin{Definition}[colbacktitle=red!75!black]{#1}{#3}#2\end{Definition}}
\crefformat{tcb@cnt@Definition}{~#2#1#3}
%
\boxiv{Axiom}{Аксиома}{yellow!5}{yellow!80!black}{tcbcolback!60!black}{tcbcolback!80!black}
\newcommand{\axm}[3]{\begin{Axiom}[colbacktitle=yellow!75!black]{#1}{#3}#2\end{Axiom}}
\crefformat{tcb@cnt@Axiom}{~#2#1#3}
%
\boxiv{Postulate}{Постулат}{blue!5}{blue!80!black}{tcbcolback!60!black}{tcbcolback!80!black}
\newcommand{\pstlt}[3]{\begin{Postulate}[colbacktitle=blue!75!black]{#1}{#3}#2\end{Postulate}}
\crefformat{tcb@cnt@Postulate}{~#2#1#3}
%
%%
%
\boxv{Question}{Вопрос}{white}{mygr}
\newcommand{\qstn}[3]{\begin{Question}{#1}{#3}#2\end{Question}}
\crefformat{tcb@cnt@Question}{~#2#1#3}
%
\boxv{Solution}{Ответ}{white}{myg!80!black}
\newcommand{\sltn}[3]{\begin{Solution}{#1}{#3}#2\end{Solution}}
\crefformat{tcb@cnt@Solution}{~#2#1#3}
%
\boxv{Formulation}{Постановка задачи}{white}{myb!80!black}
\newcommand{\fmltn}[3]{\begin{Formulation}{#1}{#3}#2\end{Formulation}}
\crefformat{tcb@cnt@Formulation}{~#2#1#3}
%
%%
%
\boxvi{History}{История}{mytheorembg}{mytheoremfr}
\newcommand{\hstr}[3]{\begin{History}{#1}{#3}#2\end{History}}
\crefformat{tcb@cnt@History}{~#2#1#3}
%
\boxvi{Story}{Поучительная история}{red!25!green!20!blue!5}{green!80!blue}
\newcommand{\lstr}[3]{\begin{Story}{#1}{#3}#2\end{Story}}
\crefformat{tcb@cnt@Story}{~#2#1#3}
%
%%
%
\boxvii{Indication}{Указание}{yellow!20!white}{yellow!80!black}{black!50!white}{3.5cm}{white}
\newcommand{\ukaz}[1]{\begin{Indication}#1\end{Indication}}
%
\boxvii{Discussion}{Обсуждение результатов}{gray!20!white}{gray!80!black}{black!50!white}{7.5cm}{white}
\newcommand{\disc}[1]{\begin{Discussion}#1\end{Discussion}}
%
\boxvii{Note}{Замечание}{red!20!white}{red!80!black}{black!50!white}{3.5cm}{white}
\newcommand{\nt}[1]{\begin{Note}#1\end{Note}}
%
\newcommand{\solve}[1]{\boxviii{Решение}{#1}}
\newcommand{\prooff}[1]{\boxviii{Доказательство}{#1}}
}
%%%%%%%%%%%%%%%
\newtcbtheorem{Titul}{Конспект лекций}{enhanced,
	before skip=2mm,after skip=2mm, 
	colback=black!5,
	colframe=black!80,
	boxrule=0.5mm,
	attach boxed title to top left={xshift=5cm,yshift*=1mm-\tcboxedtitleheight}, varwidth boxed title*=0cm,
	boxed title style={frame code={
					\path[fill=tcbcolback]
					([yshift=-1mm,xshift=-1mm]frame.north west)
					arc[start angle=0,end angle=180,radius=1mm]
					([yshift=-1mm,xshift=1mm]frame.north east)
					arc[start angle=180,end angle=0,radius=1mm];
					\path[left color=tcbcolback!60!black,right color=tcbcolback!60!black,
						middle color=tcbcolback!80!black]
					([xshift=-2mm]frame.north west) -- ([xshift=2mm]frame.north east)
					[rounded corners=1mm]-- ([xshift=1mm,yshift=-1mm]frame.north east)
					-- (frame.south east) -- (frame.south west)
					-- ([xshift=-1mm,yshift=-1mm]frame.north west)
					[sharp corners]-- cycle;
				},interior engine=empty,
		},
	fonttitle=\bfseries,
	%title={#2},#1
	}{def}
	%
	
%%%
%\boxiv{Definition}{Определение}{black!5}{black!80}{tcbcolback!60!black}{tcbcolback!80!black}
\newcommand{\titulniki}[2]{\begin{Titul}[colbacktitle=black!75]{#1}{}#2\end{Titul}}
%\crefformat{tcb@cnt@Titul}{~#2#1#3}
%
