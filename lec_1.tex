\chapter{Простейшие модели роста популяции. Основные понятия динамических систем.}
\section{Простейшие модели роста}
Пусть время дискретно, принимает целые значения и в момент времени $n$ число особей популяции равно $x_n$, а закон изменения от $n$ выражается уравнением:
$$x_{n+1} = f(x_n) \Leftrightarrow \overline{x} = f(x)$$

\subsection{Модель неограниченного роста популяции}
\exmpl{Томас Роберт Мальтус (1766 - 1839)}{Пусть количество особей в некоторой популяции в следующем поколении прямо пропорционально количеству в текущем поколении:
\[
    x_{n+1} = \lambda x_n
\]
$\lambda$ - постоянный коэффициент, определяющий темп роста.
}{}
При заданном начальном числе особей в популяции $x_0$, легко найти: 
$$ x_1 = \lambda x_0, \, x_2 = \lambda x_1 = \lambda^2 x_0, \dots, \, x_n = \lambda x_{n-1} = \lambda^n x_0$$
Пусть $x_0 > 0$ тогда возможны три принципиально разных случая поведения системы:
\begin{itemize}
    \item $\lambda > 1{:} \ \ \lim_{n \to \infty} x_n = \infty - \text{взврывообразное увеличение числа особей}$
    \item $ \lambda = 1{:} \ \ x_n = x_0 - \text{постоянная популяция}$
    \item $ 0 < \lambda < 1{:} \ \ \lim_{n \to \infty} x_n = 0 - \text{популяция вымирает}$
\end{itemize}

\subsection{Модель ограниченного роста}
\exmpl{Пьер Франсуа Ферхюльст (1804-1849)}{Пусть число особей обладает максимальным значением $M$, таким что при его достижении в следующий момент времени наступает вымирание: 
\[
    x_{n+1} = \lambda x_n(1 - \frac{x_n}{M} )
\]

$M$ - параметр аннигиляции.}{}

\begin{itemize}
    \item $x_{n} \ll M{:} $ происходит рост $x_{n+1} =  \lambda x_{n}$
    \item $x_{n} \geq M{:} $ если $x_{n+1} < 0$ или $x_{n+1} = 0$, то это трактуем как исчезновение.
\end{itemize}

\dfn{Дискретное логистическое уравнение}{
    Пусть $\frac{x_{n} }{M} = x_{n}^{\prime} $, тогда уравнение перепишется в виде:
    \[
        x_{n+1} = \lambda x_{n} (1 - x_{n}^{\prime}), \ \ x_{n}^{\prime} \in [0, 1] 
    \] 
}{} 

Модели Мальтуса и Ферхюльста наивные. В реальности есть множество внешних факторов: хищники, болезни, изменчивая доступность питания. Тем не менее они дают грубые оценки.
$\\$
$$x_0 = 0.5, \ \ \lambda \in \{0.5, 1.5, 2, 3.2, 3.5, 3.9\}$$
\begin{center}
    \begin{tabular}{c|c|c|c|c|c|c}
        n & 0.5 & 1.5 & 2 & 3.2 & 3.5 & 3.9  \\
        \hline
        1 & 0.125 & 0.375 & 0.5 & 0.8 & 0.875 & 0.5750  \\
        2 & 0.0547 & 0.352 & 0.5 & 0.512 & 0.3828 & 0.095  \\
        3 & 0.0258 & 0.342 & 0.5 & 0.799 & 0.8269 & 0.335  \\
        4 & \dag & \dag & \dag & 0.512 & 0.5009 & 0.869  \\
        5 & \dag & \dag & \dag & 0.799 & \dag & \dag  \\
        20& $1.8\cdot10^{-7}$ & 0.333 & 0.5 & 0.512 & 0.5009 & \dag 
    \end{tabular}
\end{center}
\begin{itemize}
    \item $\lambda = 0.5{:}$ вымирание
    \item $\lambda = 1.5{:}$ орбита стабилизируется в окрестности точки 0.333
    \item $\lambda = 2{:}$ неподвижная точка отображения 
    \item $\lambda = 3.2{:}$ траектория периода 2, колеблется между 0.799 и 0.512
    \item $\lambda = 3.5{:}$ траектория периода 4
    \item $\lambda = 3.9{:}$ нет закономерности, хаотическое поведение
\end{itemize}

\section{Основные понятия}
\subsection{Итерации. Понятие о каскаде.}
\dfn{Дискретная динамическая система}{
    Отображение \(\overline{x} = f(x)\) задает дискретную динамическую систему \(\{f^n\}\), где \(n \in \mathbb{Z}\) \\ если \(\overline{x} = f(x) - \text{взаимно однозначное.} \) 
}{} 
\begin{itemize}
    \item[] \(\overline{x} = f^{0}(x) \) понимаем \( \operatorname*{Id}: \overline{x} = x \)
    \item[] если \(f\) взаимно однозначное, то под \(f^{-1}\) понимаем отображение, такое что \(f(f^{-1}(x)) = x\) 
    \item[] \(k > 0: f^k(x) = f(f(\dots f(x))), \ \ f^{-k} (x) = f^{-1} (f^{-1} (\dots f^{-1}(x)))\)  
\end{itemize}

\subsection{Орбиты (траектории) динамических систем}
Всюду далее \( \mathbb{Z}, \, \mathbb{Z}_{0}^{+}, \, \mathbb{Z}_{0}^{-}.\) 
\dfn{Орбита (траектория)}{
    Орбитой (траекторией) точки \(x\) динамической системы \(\{f\}\) называется множество точек:  
    
        \[O(x) = \bigcup_{k \in \mathbb{Z}} f^k (x), \ \ \text{где}\, f - \text{взаимно однозначное}\]
        \[O(x) = \bigcup_{k \in \mathbb{Z}_{0}^{+}} f^k (x), \ \ \text{если}\, f - \text{не взаимно однозначное}\]
    }{}
\dfn{Полутраектории}{
    Для взаимно однозначных \(f\) определим положительные и отрицательные полутраектории:
    \[
        O^{+}(x) = \bigcup_{k \in \mathbb{Z}_{0}^{+}} f^k (x), \ \ O^{-}(x) = \bigcup_{k \in \mathbb{Z}_{0}^{-}} f^k (x).
    \]
    }{}
\subsection{Неподвижные точки и периодические орбиты}
\dfn{Неподвижная точка}{
    Точка \(x_0\) называется неподвижной точкой системы \(\{f\}\), если имеет место: \(f(x_0) = x_0\). 
}{}  
\dfn{Периодическая орбита}{
    Точка \(x_0\) называется периодической орбитой периода \(m > 1\), если \(f^m(x_0) = x_0\) и выполнено: 
    \[
        f^k(x_0) \not = x_0, \ \ \forall k = 1,\dots , m-1. 
    \]
}{}
\dfn{Преднеподвижная (предпериодическая) точка}{
    Точка, которая попадает в неподвижную (периодическую) точку после некоторого числа итераций называется временнонеподвижной (временнопериодической) или преднеподвижной (предпериодической).
}{}