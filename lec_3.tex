\chapter{Общие понятия теории динамических систем.}

\section{Инвариантные множества}
\dfn{}{
    \textit{Непрерывной динамической системой} или \textit{потоком} на метрическом пространстве \((X, d)\), 
    которое называется \textit{фазовым пространством} или \textit{объемлющим пространством}, называется отображение{:}
    \[
        f{:}\, X \times \mathbb{R} \longrightarrow X\]
    с групповыми свойствами:
    \begin{itemize}
        \item \(f(x, 0) = x,\, \forall x \in X\)
        \item \(f(f(x, t), s) = f(x, t + s),\, \forall x \in X, \forall s,t \in \mathbb{R}\)
    \end{itemize}
}{}

\dfn{}{
    \textit{Дискретной динамической системой} или \textit{каскадом} будем называть \(f\) удовлетворяющую
    условиям в определении выше, и если \(\mathbb{R}\) заменить на \(\mathbb{Z}\).
}{}

\noindent Везде ниже следующие обозначения:
\begin{itemize}
    \item[] для потока \(f^t(x) = f(x, t),\, t \in \mathbb{R}\)
    \item[] для каскада \(f^t(x) = f(x, t),\, t \in  \mathbb{Z}\)
\end{itemize}

\noindent Из определения 3.1 следует, что \(f^t{:}\, X \longrightarrow X\) для фиксированных \(t \in \mathbb{R}\)
или \(t \in \mathbb{Z}\) является гомеоморфизмом.
\\[1mm]
Каскад \(f^k (\text{или же}\, f^{-k}),\, k \in \mathbb{N}\) есть суперпозиция \(f^k = \underbrace{f \dots f}_{k}\, (\text{или же}\, f^{-k} = \underbrace{f^{-1} \dots f^{-1}}_{k})\). 
\\[1mm]
Поток \(f^1\) --- сдвиг на единицу времени для каскада \(f^1 = f\).
\\[1mm]
Таким образом, непрерывная (дискретная) динамическая система --- это действие гомеоморфизмами группы \(\mathbb{R}\, (\text{или же } \mathbb{Z})\) на наше топологическое пространство.

\dfn{}{
    \textit{Траекторией} или \textit{орбитой} точки \(x \in X\) называется множество 
    \[
        O_x = \{f^t(x) |\, t \in \mathbb{R} (t \in \mathbb{Z})\} \]
    траектория потока ориентируема согласно возрастанию \(t\).
}{}

\dfn{}{
    Множество \(A \subset X\) называется \textit{инвариантным множеством динамической системы}, 
    если траектория любой точки \(x \in A\) полностью принадлежит \(A\).
}{}

\dfn{}{
    Инвариантное множество \(A\) называется \textit{топологически транзитивным} (динамическая система называется \textit{топологически транзитивной} на \(A\)), 
    если \(A\) содержит всюду плотную орбиту.
}{}

\dfn{}{
    Инвариантное множество \(A \subset X\) называется \textit{локально максимальным}, если существует его открытая окрестность \(U\) такая, что{:}
    \[
        \bigcap_{t \in \mathbb{R}\, \text{или}\, \mathbb{Z}}{f^t(U)} = A\]
}{}

\noindent Заметим, что любая траектория является инвариантным множеством.
\exsz{}{
    Фазовое пространство представляется в виде объединения попарно непересекающихся траекторий динамической системы.
}{}

\noindent Выделяется 2 типа траекторий (с которыми мы на самом деле уже встречались) специального вида{:}

\dfn{}{
    Точка \(x \in X\) называется \textit{неподвижной точкой}, если \(O_x = \{x\}\). Обозначим \(Fix_{f^t}\, (\text{или}\, Fix_f)\)
    множество неподвижных точек системы \(f^t\, (\text{или}\, f)\).
}{}

\dfn{}{
    Точка \(x \in X\) называется \textit{периодической точкой} потока \(f^t\, (\text{каскада}\, f)\), если существует число \(per(x) > 0\, (per(x) \in \mathbb{N})\)
    такое, что \(f^{per(x)}(x) = x\), но \(f^t(x) \not = x\) для всех действительных (натуральных) чисел \(0 < t < per(x)\).\\[2mm] Число \(per(x)\) называется \textit{периодом периодической точки} \(x\).\\[2mm]
    В случае потока траектория периодической точки называется \textit{периодиечской траекторией} или \textit{замкнутой орбитой} и гомеоморфна единичной окружности \(\mathbb{S}^1\).
    В случае каскада траектория периодической точки называется \textit{периодической орбитой} и состоит из в точности \(per(x)\) точек. Неподвижная точка каскада --- частный случай периодической точки с периодом 1,
    для потока это неверно! \\[2mm]
    \textbf{Обозначим:} \(Per_{f^t}\, (Per_f)\) множество периодических точек системы \(f^t\, (f)\)
}{}

\noindent Под \textit{фазовым портретом динамической системы} понимают неформальное изображение фазового пространства \(X\) с некоторыми инвариантными подмножествами (неподвижные точки, периодические орбиты, инвариантные пожмножества),
дающие представление о глобальном поведении траектории динамической системы и разбиении фазового пространства на траектории.
\\[1mm]
Многие свойства динамических систем определяются асимптотическим поведением траектории \(t \to \pm \infty,\, t \in \mathbb{R}\, \text{или}\, \mathbb{Z}\){:}
