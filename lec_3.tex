\chapter{Общие понятия теории динамических систем.}

\section{Инвариантные множества}
\dfn{}{
    \textit{Непрерывной динамической системой} или \textit{потоком} на метрическом пространстве \((X, d)\), 
    которое называется \textit{фазовым пространством} или \textit{объемлющим пространством}, называется отображение{:}
    \[
        f{:}\, X \times \mathbb{R} \longrightarrow X\]
    с групповыми свойствами:
    \begin{itemize}
        \item \(f(x, 0) = x,\, \forall x \in X\)
        \item \(f(f(x, t), s) = f(x, t + s),\, \forall x \in X, \forall s,t \in \mathbb{R}\)
    \end{itemize}
}{}

\dfn{}{
    \textit{Дискретной динамической системой} или \textit{каскадом} будем называть \(f\) удовлетворяющую
    условиям в определении выше, и если \(\mathbb{R}\) заменить на \(\mathbb{Z}\).
}{}

\noindent Везде ниже следующие обозначения:
\begin{itemize}
    \item[] для потока \(f^t(x) = f(x, t),\, t \in \mathbb{R}\)
    \item[] для каскада \(f^t(x) = f(x, t),\, t \in  \mathbb{Z}\)
\end{itemize}

\noindent Из определения 3.1 следует, что \(f^t{:}\, X \longrightarrow X\) для фиксированных \(t \in \mathbb{R}\)
или \(t \in \mathbb{Z}\) является гомеоморфизмом.
\\[1mm]
Каскад \(f^k (\text{или же}\, f^{-k}),\, k \in \mathbb{N}\) есть суперпозиция \(f^k = \underbrace{f \dots f}_{k}\, (\text{или же}\, f^{-k} = \underbrace{f^{-1} \dots f^{-1}}_{k})\). 
\\[1mm]
Поток \(f^1\) --- сдвиг на единицу времени для каскада \(f^1 = f\).
\\[1mm]
Таким образом, непрерывная (дискретная) динамическая система --- это действие гомеоморфизмами группы \(\mathbb{R}\, (\text{или же } \mathbb{Z})\) на наше топологическое пространство.

\dfn{}{
    \textit{Траекторией} или \textit{орбитой} точки \(x \in X\) называется множество 
    \[
        O_x = \{f^t(x) |\, t \in \mathbb{R} (t \in \mathbb{Z})\} \]
    траектория потока ориентируема согласно возрастанию \(t\).
}{}

\dfn{}{
    Множество \(A \subset X\) называется \textit{инвариантным множеством динамической системы}, 
    если траектория любой точки \(x \in A\) полностью принадлежит \(A\).
}{}

\dfn{}{
    Инвариантное множество \(A\) называется \textit{топологически транзитивным} (динамическая система называется \textit{топологически транзитивной} на \(A\)), 
    если \(A\) содержит всюду плотную орбиту.
}{}

\dfn{}{
    Инвариантное множество \(A \subset X\) называется \textit{локально максимальным}, если существует его открытая окрестность \(U\) такая, что{:}
    \[
        \bigcap_{t \in \mathbb{R}\, \text{или}\, \mathbb{Z}}{f^t(U)} = A\]
}{}

\noindent Заметим, что любая траектория является инвариантным множеством.
\exsz{}{
    Фазовое пространство представляется в виде объединения попарно непересекающихся траекторий динамической системы.
}{}

\noindent Выделяется 2 типа траекторий (с которыми мы на самом деле уже встречались) специального вида{:}

\dfn{}{
    Точка \(x \in X\) называется \textit{неподвижной точкой}, если \(O_x = \{x\}\). Обозначим \(Fix_{f^t}\, (\text{или}\, Fix_f)\)
    множество неподвижных точек системы \(f^t\, (\text{или}\, f)\).
}{}

\dfn{}{
    Точка \(x \in X\) называется \textit{периодической точкой} потока \(f^t\, (\text{каскада}\, f)\), если существует число \(per(x) > 0\, (per(x) \in \mathbb{N})\)
    такое, что \(f^{per(x)}(x) = x\), но \(f^t(x) \not = x\) для всех действительных (натуральных) чисел \(0 < t < per(x)\).\\[2mm] Число \(per(x)\) называется \textit{периодом периодической точки} \(x\).\\[2mm]
    В случае потока траектория периодической точки называется \textit{периодиечской траекторией} или \textit{замкнутой орбитой} и гомеоморфна единичной окружности \(\mathbb{S}^1\).
    В случае каскада траектория периодической точки называется \textit{периодической орбитой} и состоит из в точности \(per(x)\) точек. Неподвижная точка каскада --- частный случай периодической точки с периодом 1,
    для потока это неверно! \\[2mm]
    \textbf{Обозначим:} \(Per_{f^t}\, (Per_f)\) множество периодических точек системы \(f^t\, (f)\)
}{}

\noindent Под \textit{фазовым портретом динамической системы} понимают неформальное изображение фазового пространства \(X\) с некоторыми инвариантными подмножествами (неподвижные точки, периодические орбиты, инвариантные пожмножества),
дающие представление о глобальном поведении траектории динамической системы и разбиении фазового пространства на траектории.
\newpage
\noindent Многие свойства динамических систем определяются асимптотическим поведением траектории \(t \to \pm \infty,\, t \in \mathbb{R}\, \text{или}\, \mathbb{Z}\){:}

\dfn{}{
    Для потока \(f^t\) (каскада \(f\)) точка \(y \in X\) называется \textit{\(\omega - \text{предельной}\) точкой для  точки \(x\)}, если 
    существует последовательность \(t_{n} \to \infty,\, t_{n} \in \mathbb{R}\, (\text{или}\, t_{n} \in \mathbb{Z})\) такая, что 
    \[
        \lim_{t_{n} \to +\infty} d(f^{t_{n}}(x), y) = 0\] 
    Множество \(\omega(x)\) всех предельных точек для \(x\) называется ее \textit{\(\omega - \text{предельным множеством}\)}.
}{}

\noindent Заменив в пределе выше \(+\infty\) на \(-\infty\) мы аналогично определим \textit{\(\alpha - \text{предельное множество}\)} \(\alpha(x)\) точки \(x\).

\clm{}{
    Если \(X\) компактно, то множество \(\omega (x)\) непусто. В частности, если система имеет неподвижную или периодическую точку \(x\), то \(\omega (x) = \alpha (x) = O_{x} \).
}{}

\noindent Множества
\[
    L_{\omega }(f^t) = cl(\bigcup_{x \in X} \omega (x))\]
\[
    L_{\alpha  }(f^t) = cl(\bigcup_{x \in X} \alpha  (x))\] 
называются \textit{\(\omega - \text{предельными}\)} и \textit{\(\alpha - \text{предельными}\)} множествами соответственно. 
\\[1mm]
Множество \(L_{f^t} = L_{\omega }(f^t)\cup L_{\alpha }(f^t)\) называется \textit{предельным множеством \(f^t\)}. 
\\[2mm]
В общем случае \(\bigcup_{x \in X} \omega (x)\) не является замкнутым. на рисунке изображен фазовый портрет потока 
на сфере \(\mathbb{S}^2\) с неподвижными точками \(A, B, C, D\), дополнение до которых состоит из замкнутых 
траектории, окружающих точки \(A, B, D\) и двух траекторий, для которых точка \(C\) является \(\omega, \alpha \)
предельной точкой (объединение этих траекторий с точкой \(C\) образует <<восьмерку>>). Дополнение до объединения \(\omega - \text{предельных}\) 
множеств есть множество точек <<восьмерки>> без точки \(C\), оно не является открытым подмножеством \(\mathbb{S}^2\). 
\\[1mm]
- - - - тут когда-то будет рисунок - - - -
\dfn{}{
    Для потока \(f^t\) (каскада \(f\)) точка \(x \in X\) называется \textit{\(\omega -\text{рекуррентной}\)}, если \(x \in \omega (x)\) 
    и называется \textit{рекуррентной}, если \(x \in \omega (x)\cup \alpha (x)\). Аналогичным образом определяется и \textit{\(\alpha -\text{рекуррентная}\)} точка. 
}{}

\noindent Рекуррентность --- возвращаемость орбиты точки в свою сколь угодно малую окрестность. Более слабый вариант возвращаемости --- \textit{неблуждаемость}. 

\dfn{}{
    Для потока \(f^t\) (каскада \(f\)) точка \(x \in X\) называется \textit{блуждающей}, если существует окрестность 
    \(U_{x} \) такая, что 
    \[
        f^t(U_{x} ) \cap U_{x} = \varnothing, \ \ \forall t > 1\, (t \in \mathbb{N})\]
    в противном случае точка \(x\) называется \textit{неблуждающей}. 
}{}

\noindent Из определений следует, что любая точка из \(U_{x} \) является блуждающей. Следовательно, множество 
блуждающих точек открыто, а множестов неблуждающих замкнуто. Множество блуждающих
точек инвариантно, так как для \(\forall t \in \mathbb{R}\, (\text{или}\, t \in \mathbb{Z})\) любая точка \(f^t(x)\) из орбиты блуждающей точки
\(x\) имеет окрестность \(U_{f^{\tau}(x)} = f^{\tau}(U_{x})\) удовлетворяет условиям определения блуждающей точки. 
\\[1mm]
Множество всех неблуждающих точек потока \(f^t\) называется \textit{неблуждающим множеством} и обозначается \(\Omega_{f^t}\). 
\\[2mm]
Заметим, что для потока на рисунке 1, неблуждающее множество совпадает с предельным \((L_{f^{t}} = \Omega_{f^t} = \mathbb{S}^2 )\). 

- - - - тут будет пример с лентой Мебиуса - - - -