\chapter{Продолжение предыдущей лекции. Топологическая классификация. Устойчивость.}

\section{Продолжение предыдущей лекции.}
Более слабый вариант, чем неблуждаемость --- тип возвращаемости связанный с \textit{\(\varepsilon\)-траекториями } 
или \textit{псевдоорбитами}. 

\dfn{}{
    \textit{\(\varepsilon - \text{сетью}\)} длины \(n\), соединяющей точку \(x\) с точкой \(y\) для \textit{каскада \(f\)}
    называется последовательность \(x = x_0,\, \dots ,\, x_{n} = y\) точек в \(M\) таких, что 
    \(d(f(x_{i-1}), x_i) < \varepsilon\) для \(i \in \{1,\, \dots,\, n\}\). 
    \\[2mm]
    \textit{\(\varepsilon - \text{сетью}\)} длины \(T\), соединяющей точку \(x\) c точкой \(y\) для \textit{потока \(f^t\)}
    называется последовательность \(x = x_0,\, \dots ,\, x_{n} = y\), для которой существует последовательность
    времен \(t_1,\, \dots,\, t_{n}\) c \(t_i \geq 1\) так, что \(d(f^{t_i}(x_{i-1}, x_i) < \varepsilon)\) для \(1 \leq i \leq  n\) и \(t_1 + \dots + t_{n} = T\). 
}{}

\dfn{}{
    Точка \(x \in X\) называется \textit{цепно-рекуррентной} для потока \(f^t\) (или каскада \(f\)), если \(\forall \varepsilon > 0 \ \ \exists T\)
    (или \(n\)), зависящее от \(\varepsilon \) и \(\varepsilon - \text{цепь}\) длины \(T\) (или \(n\)), соединяющая точку \(x\) с ней самой. 
    \\[2mm]
    Множество всех цепно-рекуррентных точек \(f^t\) (или \(f\)) называется \textit{цепно-рекуррентным множеством} \(f^t\) (или \(f\))
    и обозначается как \(R_{f^t}\) (\(R_f\) соответственно). 
}{}

\noindent Введем на \(R_{f^t}\) (или \(R_f\)) отношение эквивалентности по следующему правилу: \\
\(x \sim y \Leftrightarrow\) для любого \(\varepsilon > 0\) существует \(\varepsilon - \text{сетью}\) соединяющая точку \(x\) с точкой \(y\) и \(\varepsilon\)-цепь соединяюшая точку \(y\) с точкой \(x\). 
Две такие точки называются \textit{цепно-эквивалентными}. Класс эквивалентности называется \textit{цепной компонентой} \(R_{f^t}\) (или \(R_f\)). 
Для потоков цепная компонента \(R_{f^t}\) совпадает с компонентой связности \(R_{f^t}\). 
\\[2mm] 
Поскольку для любой \(\varepsilon - \text{окрестности}\) \(U_x\) неблуждающей точки \(x\) каскада \(f\) существует \(n \in \mathbb{N}\) такое, 
что \(f^n(U_x) \cap (U_x) \not = \emptyset\), то последовательность \(x,\, f(x),\, \dots,\, f^n(x)\) является \(\varepsilon - \text{цепью}\) длины \(n\),
соединяющей точку \(x\) c ней самой. 
\\[1mm]
Таким образом, любая неблуждающая точка является цепно-рекуррентной. Но стоит отметить, что блуждающая точка тоже может быть цепно-рекуррентной (см рисунок 3).\\
- - - - рисунок 3 - - - -
\\[2mm]
Последовательное включение инвариантных множеств: 
\[
    L_{f^t} \subseteq \Omega_{f^t} \subseteq R_{f^t} \ \ (L_{f} \subseteq \Omega_{f} \subseteq R_{f})\] 
Отличительной особенностью структурно устойчивых систем является равенство: 
\[
    L_{f^t} = \Omega_{f^t} = R_{f^t} \ \ (L_{f} = \Omega_{f} = R_{f})\]

\section{Топологическая классификация. Устойчивость.}

Качественная теория динамических систем исходит из следующего отношения эквивалентности, 
которое сохраняет разбиение фазового пространства на траектории.

\dfn{}{
    Два потока \(f^t{:}\, X \to X\) и \(g^t{:}\, X \to X\) называется \textit{топологически эквивалентными}, 
    если существует гомеоморфизм \(h{:}\, X \to X\), переводящий траектории одной системы в траектории 
    другой с сохранением ориентации на траекториях. 
}{}

\dfn{}{
    Два каскада \(f{:}\, X \to  X,\, g{:}\, X \to  X\) называется \textit{топологически сопряженным},
    если существует гомеоморфизм \(h{:}\, X \to X\) такой, что \(gh = hf\) то есть 
    диаграмма на рисунке 1.14 коммутативна. При этом гомеоморфизм \(h\) называется \textit{сопрягающим}.\\
    - - - - рисунок 1.14 - - - -
}{}

\noindent Из определения 4.4 следует, что сопрягающий гомеоморфизм переводит орбиты каскада \(f\)
в орбиты каскада \(g\). 
\\[2mm]
Непосредственная проверка топологической эквивалентности/сопряженности, как правило, является необозримой задачей. 
\\[2mm]
Некоторый объект, сохраняющийся при топологической эквивалентности/сопряженности называется \textit{топологическим инвариантом}. 
Нахождение этих инвариантов является частью задачи \textit{топологической классификации} некоторого множества \(G\) динамических систем;
под общей формулировкой которой следует понимать:
\begin{itemize}
    \item нахождение топологических инвариантов в динамической системе из \(G\)
    \item доказательство полноты множества найденных инвариантов, то есть доказательство того, что совпадение множеств топологических инвариантов является необходимым и достаточным условием топологической эквивалентности/сопряженности
    \item \textit{реализация}, то есть построение по заданному множеству топологических инвариантов стандартного представления динамических систем, принадлежащих \(G\)
\end{itemize}

\noindent Под пространством динамических система на гладком многообразии \(X\) понимают пространство \(C^\mu\) --- диффеоморфизмов \({Diff}^{r}(X)\) в случае каскада, и пространство отображений \(C^r(X\times \mathbb{R} \to X)\)
в случае потока, каждое из которых снабжено \(C^r - \text{топологией}\). Для \(r\geq 1\) каждый элемент этого пространства \(G\) называется \textit{гладкой динамической системой}. 
\\[2mm]
С любым отношением эквивалентности \(E\) на пространстве динамических систем связано определение \textit{устойчивости}. 

\dfn{}{
    Cистема \(f \in {Diff}^r(X) \, (\in C^r(X\times \mathbb{R} \to X)),\, r\geq 0\) называется \textit{\(E\)-устойчивой}, если 
    существует такая окрестность \(U(f)\) (или \(U(f^t)\)) элемента \(f\) (\(f^t\)) в \({Diff}^r(X)\) (\(C^r(X\times \mathbb{R} \to X)\)),
    что если \(\tilde{f} \in U(f)\) (\(\tilde{f^t} \in U(f^t)\)), то \(\tilde{f}\) (\(\tilde{f^t}\)) и \(f\) (\(f^t\))
    принадлежат одному и тому же классу эквивалентности \(E\). 
}{}

\noindent Понятие устойчивости для каскадов (потоков) ассоциированное с топологической сопряженностью (эквивалентностью), 
называется \textit{грубостью} (по Андронову--Понтрягину) или \textit{структурной устойчивостью} (по Пейкшото). 
В определении Понтрягина, Андронова дополнительно требуется, что при достаточной близости \(g\) к \(f\) (\(g^t\) к \(f^t\))
гомеоморфизм, осуществляющий сопряжение (эквивалентность) систем был \(C^0\)-близким к тождественному. По Пейкшото это можно
не требовать. Согласно современным представлениям структурно устойчивые и грубые динамические системы совпадают. 
\newpage
\noindent Динамические свойства системы в большой степени определяется ее поведением на неблуждающем множестве. Поэтому 
топологическая эквивалентность (сопряжение) ограничений систем на неблуждающее множество выделена в отдельное 
понятие \textit{\(\Omega - \text{эквивалентное (сопряженное)}\)}. 
Производное понятие устойчивости называется \textit{\(\Omega - \text{устойчивостью}\)}, которое слабее структурной устойчивости
 