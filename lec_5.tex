\chapter{Гиперболичность. Простейшие гиперболические множества.}

Везде в этой части рассматриваются гладкие динамические системы на гладком \(n - \text{многообразии}\, X\). 

\section{Cлучай каскада.}

\dfn{}{
    Пусть \(f_i{:}\, X \to X\) --- диффеоморфизм. Компактное \(f - \text{инвариантное}\) множество \(\Lambda \subset int(X)\)
    назвается \textit{гиперболическим}, если существует \(Df - \text{инвариантное}\) разложение касательного подрасслоения 
    \(T_{\Lambda}X\) в прямую сумму
    \[
        E_{\Lambda}^s \oplus E_{\Lambda}^u,\, x \in \Lambda \ \ (1.1.1)\]
    такое, что
    \[
        \|Df^k(v)\| \leq  c\lambda^k\|v\|,\, v \in E_{\Lambda}^s,\, k > 0\]
    \[
        \|Df^{-k}(v)\| \leq  c\lambda^k\|v\|,\, v \in E_{\Lambda}^u,\, k < 0\]
    для некоторых фиксированных \(c > 0,\, 0 < \lambda < 1\). 
}{}

\noindent Cуществование для \(f\) гиперболической структуры на \(\Lambda\)
не зависит от выбора римановой метрики. Более того, существует метрика (Ляпуновская), для которой \(c = 1\). 
\\[2mm]
Простейшим примером гиперболического множества является гиперболические неподвижные точки каскада, которые можно классифицировать
следующим образом: пусть \(f{:}\, X \to X\) --- диффеоморфизмы и \(f(p) = p\). 
Точка \(p\) является \textit{гиперболической}, когда среди собственных чисел матрицы якоби \(\frac{\partial f}{\partial x} \mid_p\) нет чисел,
по модулю равных 1. Если при этом все собственные числа по модулю меньше 1, то \(p\) называется \textit{притягивающей, стоковой точкой} или \textit{стоком},
если все собственные числа по модулю больше 1, то \(p\) называется \textit{отталкивающей, источниковой точкой} или \textit{истоком}. 
\\[2mm]
Притягивающая или отталкивающая точка называется \textit{узловой}. Гиперболическая неподвижная точка
не являющаяся узловой называется \textit{узловой точкой} или \textit{седлом}. (см. рис) \\
- - - - рисунок - - - -\\
Если точка \(p\) --- периодическая точка \(f\) с периодом \(per(p)\), то, применяя предыдущую конструкцию к диффеоморфизму \(f^{per(p)}\),
получаем классификацию гиперболических периодических точек, аналогично классификации неподвижных гиперболических точек. 
\\[2mm]
Cуществуют диффеоморфизмы, у которых всё объемлющее многообразие является гиперболическим множеством. Они были введены в динамику Д. В. Аносовым как \textit{У -- диффеоморфизмы}, 
и затем названные \textit{диффеоморфизмами Аносова}. 

\newpage

\section{Для потока.} 

С каждым гладким потоком \(f^t\) связано векторное поле \(\xi (x) = \frac{\mathrm{d}f^t}{\mathrm{d}t} \mid_{t = 0}\), множество состояний равновесия 
которого совпадает с множеством \(Fix_{f^t}\) неподвижных точек потока \(f^t\) называется \textit{гиперболическим}, если таковым является соответствующее
состояние равновесия векторного поля \(\xi\). А именно: неподвижная точка \(p\) потока \(f^t\) называется \textit{гиперболической}, 
если собственные числа матрицы якоби \(\frac{\partial \xi }{\partial x} \mid_p\) нет чисел с нулевой веществнной частью.
Если при этом все собственные имеют отрицательную часть, то точка \(p\) называется \textit{притягивающей, стоковой точкой} или \textit{стоком},
если все собственные числа имеют положительную вещественную часть, то точка \(p\) называется \textit{отталкивающей, источниковой точкой} или \textit{истоком}. 
\\[2mm]
Притягивающая или отталкивающая точка называется \textit{узловой}. Гиперболическая неподвижная точка
не являющаяся узловой называется \textit{узловой точкой} или \textit{седлом}.

\dfn{}{
    Компактное инвариантное множество \(\Lambda \subset int(X)\) потока \(f^t\), не содержащее неподвижных точек
    назвается \textit{гиперболическим}, если существует непрерывное \(Df^t - \text{инвариантное}\) разложение касательного подрасслоения 
    \(T_{\Lambda}X\) в прямую сумму
    \[
        E_{\Lambda}^s \oplus E_{\Lambda}^1 \oplus E_{\Lambda}^u,\, x \in \Lambda,\, \dim E_X^s + \dim E_X^1 + \dim E_X^u = n \ \ (1.1.2)\]
    такое, что
    \[
        \|Df^t(v)\| \leq  c\lambda^t\|v\|,\, v \in E_{\Lambda}^s,\, t > 0\]
    \[
        \|Df^{-t}(v)\| \leq  c\lambda^t\|v\|,\, v \in E_{\Lambda}^u,\, t < 0\]
    для некоторых фиксированных \(c > 0,\, 0 < \lambda < 1\), \(E_X^1\) --- одномерно и коллинеарно направлению потока. 
}{}
- - - рисунок - - - -\\

\noindent Гиперболичность замкнутой траектории потока равносильна гиперболичности неподвижной точки отображения последования (отображения Пуанкаре).
Для траекторий потока \(f^t\) из некоторой окрестности периодической орбиты \(\gamma\) существует начальная секущая \(V_{\gamma}\) и 
\textit{отображения последования} --- отображения \(\nu\), сопоставляющее точке \(v \in V_{\gamma}\) из некоторой окрестности точки 
\(p = V_{\gamma} \cap \gamma\) точку \(f^{t_0}(v)\), где \(t_0\) --- минимальное значение \(t\),
для которого \(f^t(v) \in V_{\gamma}\). 
Тогда \(p\) --- неподвижная точка диффеоморфизма \(\nu\). Траектория \(\gamma\) является \textit{гиперболической},
когда модуль собственных чисел диффеоморфизма \(\nu\) в неподвижной точке \(p\) отличен от 1. 
\\[2mm]
Асимптотическое поведение траекторий вблизи гиперболической замкнутой траектории определяется типом неподвижной 
точки \(p\) отображения последования, связанного с этой траекторией. 
\\[2mm]
Потоки, у которых всё объемлющее многообразие является гиперболическим множеством, называется \textit{потоками Аносова}
или \textit{У -- потоками}. 
\\[2mm]
\textbf{Далее все утверждения (временно) для каскадов.}
\\[2mm]
Гиперболическая структура множества \(\Lambda\) приводит к существованию у каждой точки \(x \in \Lambda\) устойчивого
\(W_x^s\) и неустойчивого \(W_x^u\) многообразий, которые в случае каскада определяется согласно следующей теореме: 

\thm{Обобщенная теорема об устойчивом многообразии}{
    (Примечание. для случая гиперболических периодических точек эта теорема называется Адамара - Перрон)\\
    Пусть \(\Lambda \subset X\) гиперболическое множество для диффеоморфизма \(f\) и \(d\) --- метрика на \(\Lambda\),
    индуцированная метрикой на \(T_{\Lambda}X\). Тогда для для любого \(x \in \Lambda\) существует устойчивое
    многообразие \(W_x^s = J_x^s(E_x^s)\), где \(J_x^s{:}\, E_x^s \to X\) --- инъективная иммерсия со следующими свойствами:
    \begin{itemize}
        \item \(W_x^s = \{y \in X {:} d(f^k(x), f^k(y)) \to 0,\, k \to +\infty\}\)
        \item если \(x, y \in \Lambda\), то \(W_x^s\) и \(W_y^s\) либо совпадают, либо не пересекаются. 
        \item \(f(W_x^s) = W_{f(x)}^s\)
        \item касательное пространство для \(W_x^s\) в точке \(y\) при \(y \in \Lambda\) есть \(E_y^s\)
        \item если \(x, y \in \Lambda\) близки, то \(W_x^s\) и \(W_y^s\) --- \(C^1 - \text{близки}\) на компактных множествах. 
    \end{itemize}
}{} 

\noindent Гладкость устойчивого (неустойчивого) многообразия \(W_x^s\) (или \(W_x^u\)) не меньше, 
чем гладкость \(f\), но из свойств инъективной иммерсии следует, что в общем случае оно является подмногообразием только локально. 
\\[2mm]
Поэтому под размерностью \(\dim W_x^s\, (\dim W_x^u)\) понимается топологическая размерность, которая в данном случае совпадают с 
\(\dim E_x^s\, (\dim E_x^u)\). В случае топологической сопряженности диффеоморфизмов \(f\) и \(f^{\prime}\) посредством 
сопрягающего гомеоморфизма \(h\) устойчивое (неустойчивое) многообразие \(W_x^s\, (W_x^u)\) преобразуется 
в инвариантное многообразие \(W_{h(x)}^s\) (или \(W_{h(x)}^u\)). Таким образом, свойство принадлежности 
точки устойчивому (неустойчивому) многообразию является топологическим инвариантом. 
\\[2mm]
Пусть \(\Lambda\) --- гиперболическое множество диффеоморфизмов \(f{:}\, X \to X\). Тогда для любого \(x \in \Lambda\) ограничение на \(TW_x^s\, (TW_x^u)\)
римановой метрики, заданной на \(TX\), индуцирует метрику \(d^s\) (или \(d^u\)) на устойчивом (неустойчивом) многообразия
\(W_x^s\, (W_x^u)\), которая называется \textit{внутренней}. 
\\[2mm]
\textbf{Филосовское замечание:}\\
Значительная часть результатов, относящихся к гиперболическим множествам, опирается на 
возможности локально <<спроектировать>> на \(W_x^s\, (W_x^u)\) свойство сжимаемости (растягиваемости)
подпространств \(E_x^s\, (E_x^u)\) относительно диффеоморфизма. Это позволяет получить локальную структуру
произведения в окрестности любой точки базисного множества благодаря существованию прямой суммы 
\(E_x^s \oplus E_x^u\), непрерывно зависящей от \(x\). 
\\[2mm]
Обозначим \(W_{x, \varepsilon}^s\) --- \(\varepsilon - \text{окрестность}\) точки \(x \in \Lambda\) на подмногообразии
\(W_x^s\) во внутренней метрике \(d^s\). 

\clm{}{
    Пусть \(\Lambda\) --- гиперболическое множество диффеоморфизма \(f{:}\, X \to X\), тогда:
    \begin{itemize}
        \item \(\forall \delta > 0\, \exists \varepsilon(\delta) > 0\) такое, что из условия \(x \in \Lambda,\, x_1, x_2 \in W_{x, \varepsilon(\delta)}^s\) следует, что \[d^s(x_1, x_2) < (1 + \delta)d(x_1, x_2)\]
        \item \(\exists \varepsilon > 0,\, \mu < 1\) такие, что из условия \(x \in \Lambda,\, x_1, x_2 \in W_{x, \varepsilon}^s\) следует, что \[d^s(f(x_1), f(x_2)) < \mu d^s(x_1, x_2)\]. 
    \end{itemize}
}{}